\chapter{Market Impact \& Optimal Execution}
\section{Statistics of the order book}
Bouchaud et al. \cite{BouchaudMezardPotters2002} gives a model of the
order book: let $a(t)$ be the lowest asking price and $b(t)$ the
highest bidding price at time $t$. The new buy limt order comes in at
price $b(t) - \Delta_b$ and the new sell limit order comes in at
$a(t) + \Delta_a$. The authors reported that $\Delta_a, \Delta_b$
could be considered independent, identically distributed. Let
$\Delta$ have the same distribution as $\Delta_a$ and $\Delta_b$.

The distribution of $\Delta$ has Pareto tail. Let
$f_{\Delta | \Delta \geq 1}$ denote the probability density function
of $\Delta$ conditional on $\Delta \geq 1$.
\begin{equation}
  \label{eq:rtwhyt}
  f_{\Delta | \Delta \geq 1}(x) = {
    K^\alpha
    \over
    (\mu + x)^{\alpha + 1}
  } \frac{1}{\P(\Delta \geq 1)}
\end{equation}
The tail index $\alpha$ is reported to be approximately $0.6$ for all
equities. Note this implies $\E \Delta = \infty$. It is also reported
that the distribution of $\log V$, $V$ being the size of the new limit
order, buy or sell, is uniformly distributed. Formally
\begin{equation}
  \label{eq:pojfg}
  \P(V \leq x) = a(\log x - b)
  \quad
  e^b \leq x \leq e^{1/a + b}
\end{equation}
The authors also reported, empirically
\begin{eqnarray}
  \E(V|\Delta = x, x \leq \Delta^*) &=& c \nonumber \\
  \E(V|\Delta = x, x > \Delta^*) &=& c' x^{-\nu},\; \nu \approx 1.5 \label{eq:rfepomj}
\end{eqnarray}
where $\Delta^*$ is a constant dependent on the stock itself.

From \eqref{eq:rtwhyt} and \eqref{eq:rfepomj}, one can derive the average
volume when $\Delta \leq \Delta^*$ and when $\Delta > \Delta^*$:
\begin{eqnarray*}
  \E(V \1{\Delta \leq \Delta^*}) &=& c \P(\Delta \leq \Delta^*) \\
  \E(V \1{\Delta > \Delta^*}) &=& \int_{\Delta^*}^\infty c' x^{
                                  -\alpha - \nu - 1
                                  } K^\alpha \left(
                                  \frac{\mu}{x} + 1
                                  \right)^{-\alpha - 1}
                                  \frac{dx}{\P(\Delta > 1)}
\end{eqnarray*}
By Karamata's theorem,
\[
  \E(V \1{\Delta > \Delta^*}) \sim c' \left(
    {\mu \over \Delta^*} + 1
  \right)^{-\alpha - 1} K^\alpha
  {(\Delta^*)^{-\alpha-\nu} \over \alpha + \nu}
  {1 \over \P(\Delta > 1)}
  \quad
  \text{ as }
  \Delta^* \to \infty
\]


\section{Square root law}
Bouchaud et al. \cite{Bouchaud2011} gave a model to the average price
change of an equity following the completion of a large order. It was
reported that the law was valid regardless of the way of execution -
over-the-counter or electronically, and regardless of the execution
strategy - using limit or market orders. Assume the order is executed
in its entirety within 1 trading day.
\begin{itemize}
\item $\{P_t\}_{t \in \reals_+}$: the stochastic process of prices of
  the equity.
\item $\tau$: time of the trade that fullfils the order. The first
  trade of the order is assumed to happen at a deterministic time
  $s < \tau$.
\item $Q$: size of the order, i.e. the number of shares that are
  intended to be traded - bought or sold.
\item $\{V_n\}_{n \in \naturals}$: Daily trading volumes of the
  equity.
\item $\{\sigma_n\}_{n \in \naturals}$: Daily volatilities of the
  equity.
\end{itemize}
Bouchaud et al. \cite{Bouchaud2011} argued
\[
  \E (P_\tau - P_s) = c \E \sigma_{\ceil \tau} \left(
    {Q \over \E V_{\ceil \tau}}
  \right)^{1/2}
\]
The main arguments are as follows: Traders do not reveal their
intensions too early but rather send orders depending on the market
situation. Hence there is a latent order book that reflect the
interests of traders. Let
\begin{itemize}
\item $\lambda(u)$: the number of new shares added to the buy side of
  the latent order book at price $P_t - u$ in unit time and equally,
  the number of new shares added to the sell side of the latent order
  book at price $P_t + u$ in unit time. It is assumed that the buy
  side and the sell side of the latent order book change in the same
  way with the market price and $\Delta$ is independent of the market
  price.

\item $v(u, u')$: the number of shares on the buy side of the latent
  order book that are moved from price $P_t - u$ to price $P_t - u'$
  in unit time. Equally, it is also the number of shares on the sell
  side of the latent order book that are moved from price $P_t + u$
  to price $P_t + u'$.

\item $W (u, t)$: the volume on the latent order book for buy (sell)
  at price $P_t - u$ ($P_t + u$). $\rho(u, t) = \E W(u, t)$.
\end{itemize}

Essentially we assume that changes to the latent order book are always
symmetric and speculative in nature.


\section{Capacity management}
There have been several authors who contributed to the literature of
capacity management. In the asset management literature,
Cantara et al. \cite{Cantara2016} treated capacity management as a
risk measure but stopped short of providing a formula for estimating
the capacity of a given equity. Naik et al. \cite{EUFM:EUFM353}
investigated capacity constraints and alpha attenuation by looking
into the influence on alpha of capital flowing into an investment
strategy in different time periods. The investment strategies
considered in their studies included {\it security selection}, {\it
  macro}, {\it directional traders}, {\it multi-process}, etc. They
analysed data from hedge funds and each hedge fund is categirised into
a certain strategy. When a positive capital in-flow to a strategy
actually leads to a decreased alpha, capacity constraint of that
strategy is considered to have been reached.

% The problem that we want to answer is this: given an equity that we
% would like to trade, what is the capacity of the equity? In other
% words, what is the maximum capital that can be traded on this equity
% on a daily basis without making a significant market impact?

We approach the problem from a different angle. Instead of considering
the capacity of a strategy, which in the terminology of Naik et
al. \cite{EUFM:EUFM353} can be a selection of equities, we consider
the capacity of an equity; and instead of detecting whether the
capacity constraint has been reached, we provide a method to estimate
the capacity, which a fund manager can take into account when
allocating his assets.

We want to answer this question: What is the maximum number of shares
that a fund manager can trade on a day-to-day basis without
making a significant market impact and hence reducing alpha?

For liquid equities, a homogeneous Poisson process is a good
description of the counting process of the number of trades. Formally,
let $N_t(s)$, $t \in \naturals, s \in [0, 1]$ be the number of trades
recorded up to time $s$ on day $t$. We use 1 trading day as the unit
of time, so $N_t(0) = 0$ and $N_t(1)$ is the total number of trades
recorded on day $t$.

Let $T_{t, k}$ be the arrival time of the $k$-th trade on day $t$ such
that $N_t(s) = \sum_{k=1}^\infty \1{T_{t, k} \leq s}$.
We assume $T_{t, k} - T_{t, k-1} = \tau_{t,k} \sim
\text{Exp}(c_N \Lambda_t)$. Then $\E(\tau_{t, k} | \Lambda_t) =
1/c_N \Lambda_t$. $c \Lambda_t$ is the intensity of the Poisson process
modeling $N_t(\cdot)$. $c_N$ is a constant and we assume $\Lambda_t \sim
\text{IG}(\alpha, 1/\alpha)$, i.e.
\[
  \frac{d}{dx} \P(\Lambda_t \leq x)
  =
  {1 \over \alpha^\alpha \Gamma(\alpha)}
  {e^{-1/\alpha x} \over x^{\alpha + 1}}
\]

%% Built on the result of Bouchaud et al.,
\subsection{A market without market makers}
In such a symmetric market,  we may model the daily return $X_t = P_t - P_{t-1}$ as
\begin{equation}
  \label{eq:kopr}
  X_t
  =
  \sum_{k=1}^{N_t(1)} X_{t,k}
  =
  \left( \sum_{k=1}^{N_t(1)} Q_{t,k} \right)^{-1/2}
  \sum_{k=1}^{N_t(1)}
  \sigma_t
  \sqrt{
    \Lambda_t Q_{t,k}
  } Z_{t, k}
\end{equation}
where $X_{t,k}$ is the return of the $k$th trade on day $t$;
$Q_{t,k}$ is the number of shares exchanged on this trade and
$Z_{t,k}$ are iid standard normal variables. $\sigma_t$ is a parameter
dependent on the equity and the time $t$. By Martingale pricing theory
we must have $\E X_{t,k} = 0$ for all $t$ and $k$. Clearly,
\[
  {X_t \over \sigma_t}
  \left|
    N_t(1), \Lambda_t, \{Q_{t,k}\}_{k=1}^{N_t(1)}
  \right. \sim N(0, \Lambda_t),
\]
then it follows
\begin{equation}
  \label{eq:jirtfg}
  {X_t \over \sigma_t} \sim t(2\alpha)
\end{equation}
Equation \eqref{eq:kopr} also allows us to estimate the price change
caused by a single trade, i.e. the market impact of the trade.
Let $V_t = \sum_{k=1}^{N_t(1)} Q_{t, k}$, then
\begin{equation}
  \label{eq:oocrvfe}
  {X_{t, k} \over \sigma_t}
  \sqrt{V_t \over Q_{t,k}}
  =
  \Lambda_t^{1/2} Z_{t,k}
  \sim
  t(2\alpha)
\end{equation}

\subsection{A market with market makers}
Market makers provide liquidity to the market and in return enjoy the
priviledge of seeing incoming orders first and in their entirety. For
an ordinary trader who doesn't have this priviledge and accesses the
market via an market maker, the description of the slippage offered
by equation \eqref{eq:oocrvfe} is insufficient.

To capture the information asymmetry in a market mediated by market
makers, we propose the following model:
\begin{equation}
  \label{eq:kopr1}
  X_t
  =
  \sum_{k=1}^{N_t(1)} X_{t,k}
  =
  \sum_{k=1}^{N_t(1)}
  \sigma_t \sqrt{
    Q_{t,k} \over V_t
  }
  \left(
    \sqrt \Lambda_t  Z_{t, k}
    + \xi B_{t, k}
  \right)
\end{equation}
where $V_t = \sum_{k=1}^{N_t(1)} Q_{t,k}$ is the total volume traded
on day $t$,
\[
  B_{t,k} = \left\{
    \begin{array}{ll}
      1 & \text{A common trader buys from a market maker} \\
      0 & \text{A common trader buys from another common trader} \\
      -1 & \text{A common trader sells to a market maker}
    \end{array}
  \right..
\]
where $\xi$ is a constant. We assume the following discrete distribution for $B_{t,k}$:
\[
  \P(B_{t,k} = 1) = \P(B_{t,k} = -1) = p,
  \quad
  \P(B_{t,k} = 0) = 1 - 2p
\]
Bouchaud et al. \cite{Bouchaud2011} estimated
this constant to be of order 1. But in fact, intraday trading data of
liquid US equities suggest that the effect caused by the term $\xi
B_{t,k}$ is rather small. Table \ref{tab:spy_big_trades} shows the
records of some big trades in October and September of 2017. 
\begin{table}[htb!]
  \begin{tiny}
    \centering
    \begin{tabular}{|c|c|c|c|c|c|c|r|r|r|c|c|}
      \hline
      day & $t_0$ & $t_1$ & $t_2$ & $P_0$ & $P_1$ & $P_2$ & Volume & $\log {P_1 \over P_0}$ & $\log {P_2 \over P_1}$ & $\Delta P$ & $\overline{\Delta P}$\\
      \hline
      2017-10-05 & 11:28:49 & 11:28:50 & 11:28:51 & 253.899198 & 253.890001 & 253.890000 & 1586300 & -0.36 & -0.00 & -0.9 & 3.50\\
      2017-09-27 & 15:52:27 & 15:52:28 & 15:52:29 & 250.150000 & 250.140000 & 250.150000 & 1203180 & -0.40 & 0.40 & -1.0 & 3.10\\
      2017-10-04 & 10:38:17 & 10:38:20 & 10:38:21 & 252.750000 & 252.749845 & 252.740000 & 977844 & -0.01 & -0.39 & -0.0 & 3.35\\
      2017-09-26 & 12:47:10 & 12:47:11 & 12:47:22 & 249.070000 & 249.130000 & 249.070000 & 963352 & 2.41 & -2.41 & 6.0 & 3.45\\
      2017-10-03 & 10:05:35 & 10:05:38 & 10:05:39 & 252.480000 & 252.470000 & 252.473333 & 900809 & -0.40 & 0.13 & -1.0 & 2.83\\
      2017-09-27 & 15:50:00 & 15:50:01 & 15:50:02 & 250.180803 & 250.167839 & 250.139939 & 535878 & -0.52 & -1.12 & -1.3 & 2.07\\
      2017-09-27 & 15:04:57 & 15:04:59 & 15:05:00 & 250.320178 & 250.359544 & 250.320000 & 514387 & 1.57 & -1.58 & 3.9 & 2.03\\
      2017-09-27 & 14:13:21 & 14:13:23 & 14:13:24 & 249.980000 & 250.014426 & 250.025289 & 493656 & 1.38 & 0.43 & 3.4 & 1.98\\
      2017-10-02 & 15:15:51 & 15:15:52 & 15:15:53 & 252.120000 & 252.056998 & 252.039569 & 474800 & -2.50 & -0.69 & -6.3 & 2.23\\
      2017-09-29 & 13:31:01 & 13:31:02 & 13:31:03 & 251.060000 & 251.059843 & 251.052171 & 406400 & -0.01 & -0.31 & -0.0 & 1.87\\
      2017-10-06 & 10:38:45 & 10:38:47 & 10:38:48 & 254.092173 & 254.090000 & 254.100000 & 402699 & -0.09 & 0.39 & -0.2 & 1.86\\
      2017-10-16 & 14:51:10 & 14:51:11 & 14:51:13 & 255.110000 & 255.100000 & 255.100000 & 400000 & -0.39 & 0.00 & -1.0 & 2.17\\
      2017-09-01 & 11:15:41 & 11:15:42 & 11:15:43 & 248.130000 & 248.090010 & 248.130000 & 390427 & -1.61 & 1.61 & -4.0 & 2.00\\
      2017-09-22 & 15:59:56 & 15:59:57 & 15:59:58 & 249.511926 & 249.483784 & 249.432256 & 387672 & -1.13 & -2.07 & -2.8 & 2.02\\
      2017-09-18 & 10:10:39 & 10:10:40 & 10:10:41 & 249.951000 & 249.997537 & 249.960000 & 376029 & 1.86 & -1.50 & 4.7 & 2.11\\
      2017-10-03 & 14:58:24 & 14:58:25 & 14:58:27 & 252.690000 & 252.690000 & 252.690000 & 357866 & 0.00 & 0.00 & 0.0 & 1.79\\
      2017-09-28 & 12:23:27 & 12:23:32 & 12:23:33 & 250.260714 & 250.231762 & 250.260000 & 328921 & -1.16 & 1.13 & -2.9 & 2.20\\
      2017-10-11 & 11:05:07 & 11:05:09 & 11:05:15 & 254.605000 & 254.600000 & 254.606000 & 312000 & -0.20 & 0.24 & -0.5 & 1.90\\
      2017-10-03 & 10:34:46 & 10:34:47 & 10:34:48 & 252.390000 & 252.399989 & 252.380000 & 300319 & 0.40 & -0.79 & 1.0 & 1.63\\
      \hline
    \end{tabular}
    \caption{\scriptsize Big trades of SPY. $P_0$: price of the trade immediately before to the big trade;
      $P_1$: price of the big trade; $P_2$: price immediately after the big trade.
      The returns $\log {P_1 \over P_0}$ and $\log {P_2 \over P_1}$ are scaled up by $10^4$ for the convenience of
      display. The prices are trading prices averaged over the same second. $\Delta P = P_1 - P_0$ is noted in cents.
      $\overline{\Delta P}$ is the predicted upper bound in cents with 60\% confidence level.
    }
    \label{tab:spy_big_trades}
  \end{tiny}
\end{table}
To estimate the parameters of $\xi$ and $p$, one can use a maximum-likelihood estimation
(MLE) procedure. Let 
\[
  Y_{t, k} = \Lambda_t^{1/2}  Z_{t, k} + \xi B_{t, k}
\]
For convenience, we use $\psi_{\nu}(\cdot)$ to denote the density function of $t(\nu)$ distribution. By conditioning
on $B_{t,k}$ and $\Lambda_t$, one can find the density function of $Y_{t,k}$. Call it $f_Y$. We have
\begin{equation}
  \label{eq:tghjporf}
  f_Y(y) = p [\psi_{2\alpha}(y - \xi) + \psi_{2\alpha}(y + \xi)] + (1 - 2p) \psi_{2\alpha}(y)  
\end{equation}

% For the purpose of estimation
% we will use intraday data, but the parameters obtained thereby vary slowly and need to be re-estimated only
% at time intervals much longer than a day.

% Consider trading days with extremely large positive or negative returns. These days are characterised by major
% news penetrating the market and a consistent view of the market being shared by all traders. Hence the trading
% is dominently between a common trader and a market maker. In a bull market of this kind, we can expect most of
% $\{B_{t,k}\}_{k=1}^{N(t)}$ to be +1; and in a bear market, we can expect most of $\{B_{t,k}\}_{k=1}^{N(t)}$ to be -1.

% For simplicity, we assume all $\{B_{t,k}\}_{k=1}^{N_t(1)}$ are +1 in a bull market. Let
% \[
%   X'_t = \xi \sigma_t \sum_{k=1}^{N_t(1)} \sqrt{Q_{t, k} \over V_t} \Lambda_t^{-1/2} 
% \]
% We have
% \[
%   \E\left(
%     {X_t \over \sigma_t} \left| B_{t,k}=1, k=1,\dots N_t(1) \right.
%   \right) = \E\left( {X'_t \over \sigma_t} \right)
% \]
% It follows
% \begin{equation}
%   \label{eq:rtyhdfd}
%   \E \left(
%     {X'_t \over \sigma_t} \left| N_t(1), V_t, \{B_{t,k}\}_{k=1}^{N_t(1)} \right.
%     \right)
%     =
%     \xi \sum_{k=1}^{N_t(1)} \sqrt{Q_{t,k} \over V_t} \E \Lambda_t^{-1/2}
% \end{equation}
% $1/\Lambda_t \sim G(\alpha, 1/\alpha)$, so
% \[
%   \E \Lambda_t^{-1/2} = m_G(1/2) = {\Gamma(\alpha + 3/2) \over \Gamma(\alpha)} \alpha^{3/2}
% \]
% where $m_G(\cdot)$ denotes the moment generating function of $G(\alpha, 1/\alpha)$ distribution. Equation
% \eqref{eq:rtyhdfd} gives
% \[
%   \E \left(
%     {X'_t \over \sigma_t}
%   \right) = \xi m_G(1/2) \E \left[
%     \sum_{k=1}^{N_t(1)}
%       \sqrt{Q_{t,k} \over V_t}
%   \right]
% \]
% Assume $\left\{\sqrt{Q_{t,k} \over V_t}\right\}_{t \in \naturals, 1 \leq k \leq N_t(1)}$ is an iid array and each
% $\sqrt{Q_{t,k} \over V_t}$ is independent of $N_t(1)$. Let $\kappa = \E \sqrt{Q_{t,k} \over V_t}$.
% Moreover,
% \[
%   \E N_t = c_N \E \Lambda_t = {c_N \over \alpha (\alpha - 1)}
% \]
% Finally we have
% \[
%   \E\left(
%     {X_t \over \sigma_t} \left| B_{t,k}=1, k=1,\dots N_t(1) \right.
%   \right)
%   =
%   \E \left( {X'_t \over \sigma_t} \right)
%   =
%   \xi \kappa c_N
%   {\Gamma(\alpha + 3/2) \over \Gamma(\alpha + 1)}
%   {\alpha^{1/2} \over \alpha - 1}
% \]
% By the strong law of large numbers
% \[
%   {1 \over |I|} \sum_{t \in I} {X_t \over \sigma_t}
%   \overset{a.s.}{\to}
%   \E\left(
%     {X_t \over \sigma_t} \left| B_{t,k}=1, k=1,\dots N_t(1) \right.
%   \right),
%   \quad
%   \text{ as }
%   |I| \to \infty
% \]
% where $I = \{1 \leq t \leq n: 1 \leq \forall k \leq N_t(1), B_{t, k} = 1\}$ and
% \[
%   |I| = \sum_{t=1}^n \1{t \in I}.
% \]
% Here $\{1, 2, \dots, n\}$ correspond to the days included in one's sample.
% Clearly $|I| \to \infty$ as $n \to \infty$. Similarly,
% \[
%   {1 \over n} \sum_{t=1}^n N_t \overset{a.s.}{\to} \E N_t = {c_N \over \alpha (\alpha - 1)},
%   \quad
%   {1 \over n} \sum_{t=1}^n \sqrt{Q_{t, k} \over V_t} \overset{a.s.}{\to} \E \sqrt{Q_{t, k} \over V_t} = \kappa,
%   \quad
%   \text{ as } n \to \infty
% \]

\section{SPY - an ETF following S\&P 500}
\subsection{Model of the volatility}
In view of equation \eqref{eq:jirtfg}, we fit a GARCH($p, q$) model
with $t(2\alpha)$-innovations to the daily returns of  SPY from
2010-10-15 to 2017-10-19. Based on information criteria, we choose a
GARCH($1, 1$) model:
\[
  \sigma_t^2 = \omega + \alpha_1 X_{t-1}^2 + \beta_1 \sigma_{t-1}^2
\]
The parameter values are estimated to be
\[
  \omega = 2.690660\times 10^{-6}, \alpha_1 = 0.1692482,
  \beta_1 = 0.8098616, \alpha=2.722758
\]
The Bayesian information criterion (BIC) value of this mode is
-6.907569. The alternative GARCH($2,1$) and GARCH($1, 2$) models have
BIC values -6.905943 and -6.903336, respectively.

\subsection{Model of the volume}
From daily trading data of SPY, an ARIMA($1,1,1$) model for
$\{\log V_t\}_{t \in\naturals}$ suggests itself. Let $W_t = \log V_t - \log
V_{t-1}$.
\[
  W_t = \phi W_{t-1} + \theta \epsilon_{t-1} + \epsilon_t
\]
where $\{\epsilon_t\}_{t \in \naturals}$ is an iid sequence of
random variables. The parameter values are estimated as follows:
\[
  \phi = 0.3345237, \theta = -0.8830647
\]
The distribution of $\{\epsilon_t\}_{t \in \naturals}$ is found to be
best described by a non-central $t$-distribution, whose density
function is
\begin{equation}
  \label{eq:h6yhj}
  {d \over dx} \P(\epsilon_t \leq x)
  =
  {\Gamma({\nu/2 + 1/2}) \over \Gamma({\nu/2})}
  {1 \over \sqrt{\nu \pi}}
  \left[
    1 + {(x - m)^2 \over s^2 \nu}
  \right]^{-\nu/2 - 1/2},
  \quad \forall t \in \naturals
\end{equation}

\subsection{Parameter Estimation}
With intraday data from 2017-08-30 to 2017-10-16, we have estimated
the parameters as follows, using MLE and the density function $f_Y$ as
in equation \eqref{eq:tghjporf}:
\[
  \xi = 1.0 \times 10^{-6},
  \quad
  p = 0.01
\]

\subsection{Slippage \& Capacity}
Equation \eqref{eq:kopr1} allows a fund manager to control the
slippage associated with his trade. He wants to limit the size of his
trade so that with probability $1 - \delta$, the price rise/fall
caused by his buying/selling the equity is smaller than a specified
value $x$. Let $F_t(\cdot, \nu)$ denote the distribution function of
$t$-distribution of $\nu$ degrees of freedom. For a ``buy''-trade, the
fund manager wants to choose $Q_{t,k}$ such that for a given $x$
\[
  \P(X_{t,k} > x) \leq \delta
\]
Equation \eqref{eq:kopr1} gives
\begin{align*}
  \delta &\geq
           1 - F_t\left(
           {x - \xi \over \sigma_t} \sqrt{V_t \over Q_{t,k}}, 2\alpha
           \right) \\
  Q_{t, k} &\leq
             {V_t  (x - \xi)^2 \over \sigma_t^2}
             \left[
             F_t^{-1}(1 - \delta, 2\alpha)
             \right]^{-2}
\end{align*}
The right side of the last equation gives an upper limit to the number
of shares that can be traded without siginificantly affecting the price.
Here the ``significance'' is parameterized by $x$ and $\delta$, and so is
the upper limit $Q_{t,k}$.



% It may be more informative to tabulate the upper bound on the slippage
% for a given confidence level, a given ratio of $Q_{t,k}/V_t$, and a
% fixed price. For convenience, we fix the price to \$100. The upper
% bounds are shown in the following table.
% \begin{table}[htb!]
%   \centering
%   \begin{tabular}{|c|c|c|c|c|c|c|c|c|c|c|}
%     \hline
%     \multicolumn{2}{|c}{ } & \multicolumn{9}{c|}{confidence level} \\
%     \hline
%     & & 10\% & 20\% & 30\% & 40\% & 50\% & 60\% & 70\% & 80\% & 90\% \\
%     \hline
%     \multirow{4}{*}{${Q_{t,k}\over V_t}$} & 0.1\% & -1.91 & -1.20 & -0.73 & -0.35 & 0.00 & 0.35 & 0.73 & 1.20 & 1.91 \\
%     & 0.2\% & -2.70 & -1.69 & -1.03 & -0.49 & 0.00 & 0.49 & 1.03 & 1.69 & 2.70 \\
%     & 0.3\% & -3.31 & -2.07 & -1.26 & -0.60 & 0.00 & 0.60 & 1.26 & 2.07 & 3.31 \\
%     & 0.4\% & -3.82 & -2.39 & -1.46 & -0.70 & 0.00 & 0.70 & 1.46 & 2.39 & 3.82 \\
%     \hline
%   \end{tabular}
%   \caption{Slippage upper bounds (cents)}
% \end{table}



% Estimation of $\E V$ is straightforward. As for
% $\sigma$, an option is to employ a GARCH($p, q$) model, if only daily
% data are available or feasible to process. That is
% \[
%   \sigma_{t}^2 = \omega + \sum_{i=1}^p \alpha_i X_{t-i}^2
%   + \sum_{i=1}^q \beta_i \sigma_{t-i}^2
% \]
% where $p, q$ are respectively the ARCH and GARCH orders of the model;
% $\omega, \{\alpha_i\}, \{\beta_i\}$ are constants estimated using
% daily data. Other possible models include the exponential GARCH
% (EGARCH), GJR-GARCH which accounts for the leverage effect, as well as
% stochastic volatility (SV) models. SV models' estimates are generally
% more accurate, particularly when it is possible to use intraday data.

% The availability of intraday data as well as the computational
% resources one is willing to invest on processing them makes a big
% difference. We investigate these two scenarios separately.






