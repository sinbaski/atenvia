\documentclass{article}
\RequirePackage[colorlinks,citecolor=blue,urlcolor=blue,bookmarksopen=true]{hyperref}
\usepackage{bookmark}
\input{../physics_common}

\title{Hedge Algorithms}
\author{Xiaolei Xie \\
  xiaolei.xie@atenvia.com
}
\date{\today}

\begin{document}
\maketitle

\section{A Linear Model}\label{sec:lm}
At the centre of the hedge algorithms described in this report are
linear models of returns. Let $Y_t$ denote the return of the
portfolio at time $t$, and $X_t$ denote the return of SPY at time $t$.
We want to hedge the portfolio against fluctuations of SPY, i.e. make
the returns of the portfolio uncorrelated (neutral) to the returns of SPY,
when such neutrality is desirable, for example, when the market is considered
to be trending down.

The idea is to assume a persistent linear relation between the returns of
the portfolio and the returns of SPY. In mathematical terms, we write
\begin{equation}
  \label{eq:ghtht}
  Y_t = a X_t + \epsilon_t  
\end{equation}
where $a$ is the regression coefficient and $\epsilon_t$ is a random
variable independent of $X_t$.
Viewed from a different angle, equation \eqref{eq:ghtht} gives a hedging
algorithm. Moving the $a X_t$ term to the left side, we get
\[
  Y_t - a X_t = \epsilon_t    
\]
This formula says, if, for every dollar worth of the portfolio, we short
$a$ dollars worth of SPY, the return of our holding, i.e. the portfolio
plus the short position on SPY, is a random variable independent of
$X_t$ -- we now hold a collection of assets whose overall return is
uncorrelated to that of SPY, which is what we set off to achieve.

To estimate the value of $a$ and hence the correct number of shares of SPY
to short, one can use the procedure of {\it Ordinary Least Squares} (OLS).
We cover this topic in \S\ref{sec:ols}. Once the value of $a$ has been estimated,
it remains to determine when a short position on SPY should be
attained. Algorithm 1 (cf. \S\ref{sec:when}) bases this decision on
the drawdown of the portfolio while the others rely on estimating
$X_t$ and assessing its significance. The details are found in \S\ref{sec:when}.

% In the case of the latter, a
% short position on SPY should be in place on day $t$ when the return of SPY 
% on this day ($X_t$) is negative. But we will never know for sure a future return
% of SPY, so we have to be content with an estimate.

\section{Timing}\label{sec:when}
In this section we describe the hedge algorithms one by one. They all
share the procedure of computing $a$, the correct dollar amount of SPY
to hedge, whenever such a hedge is relevant, and differentiate from
each other in their ways to assess the significance of a possible
downward movement of SPY. Each way of assessment leads to an answer to
the question of ``when to hedge'' and thus to a distinctive hedge
algorithm.

The first hedge algorithm triggers the hedge whenever the
drawdown of the portfolio exceeds a pre-defined level. The argument
and the pre-condition for this is that the return of the portfolio
($X_t$) is positively correlated to that of SPY ($Y_t$), owing to
the fact that the portfolio largely consists of long positions of
equities that benefit from a bullish market. Hence when the
portfolio is losing value, a downward-trending SPY is likely to be
responsible.

By similar reasoning, we lift the hedge, i.e. clear the short position
on SPY, when the portfolio drawdown is below a pre-defined level. We
consider a significant reduction in the drawdown as an indication that
the SPY has reversed its previous negative trend and made a short
position on it undesirable. More precisely, we short SPY if we find
\[
  D_t = 1 - {P_t \over \max_{s \leq t} P_s} > c_1 > 0
\]
where $P_s$ denotes the price of SPY at time $s$. $c_1$ is a positive
constant. To avoid alternating behaviour in hedging, we don't clear
the short position on SPY as soon as the drawdown $D_t$ falls below
$c_1$. Rather, we keep the position until the drawdown has receded
below a value even lower than $c_1$. To make this precise, let $a_t$
denote the size of the hedge position on day $t$. Let $f(t)$ be the
function that gives the size of the hedge position on day $t$ using
OLS according to the descriptions in \S \ref{sec:lm}. We short SPY by
$a_t = f(t)$ dollars if $D_t > c_1$; we clear the
hedge position by setting $a_t = 0$ if $D_t < c_2 < c_1$.


Moreover, we would like to minimise the trading cost attributed to
hedging. For this purpose, we keep the size of the hedge position
unchanged from the previous day as long as the dollar amount of
adjustment to the position is relatively small. In precise terms, if
$a_{t-1} > 0$, then we assign $a_{t-1}$ to $a_t$ unless
$|f(t)/f(t-1) - 1| > b > 0$.

The 2nd, 3rd and 4th algorithms trigger the hedge by estimating
the mean of the return of SPY ($X_t$) and assessing its
significance. While the procedure of estimation is straightforward and
unsophisticated, assessing the significance is more complicated. On
the one hand, we want to be reasonably certain that a negative trend
is at work before taking any action; on the other hand, we want to be
able to respond to a negative trend at an early stage. These are
indeed contradicting wishes and we have to strike a balance.

An obvious way to estimate $X_t$ is to assume that the same law of
distribution persists over its recent history up to the day $t$. Let
$L$ denote the length of this history as measured by days. We estimate
the distribution of $(X_s)_{s=t-L, \dots, t-1}$ and assume the same
distribution applies to $X_t$. For a small $L$, i.e. a short history,
the effect of heteroscedasticity is negligible. So we assume, for
simplicity, that $(X_s)_{s=t-L, \dots, t-1}$ are independent and
identically distributed. Under this assumption, estimating the mean of
$X_t$ ($\E X_t$) is straightforward. By the law of large numbers,
\[
  \bar X_t = {1 \over L} \sum_{i=t-L}^{t-1} X_i \overset{a.s.}{\to} \E X_t,
  \quad
  L \to \infty
\]
That is, the arithmetic average of the historical returns tends to the
mean value of $X_t$ as the length of the history taken into account
tends to infinity. However, an erratic behaviour will result if we
short SPY as soon as a negative $\bar X_t$ is obtained, regardless of
its magnitude. Therefore, we must assess the significance of the
estimated value of $\E X_t$ and act only if this value is significant.
\begin{enumerate}
\item The 2nd algorithm assesses the significance of the mean of $X_t$
  by comparing $\bar X_t$ with a constant threshold $c_1 > 0$. To make this
  threshold more intuitive, we annualise $\bar X_t$. On average, there
  are 252 trading days in a year. If $252 \bar X_t < -c_1 < 0$, we
  consider the estimated downward trend of SPY to be significant and
  certain enough to support a short position.

  Conversely, if we find $252 \bar X_t > c_2 > 0$, we consider SPY to be
  experiencing a significant upward trend and proceed to clear any
  short position on SPY, i.e. to lift the hedge if it is in place.

  In all other situations, we keep the hedge position unchanged from
  the previous day.
  
\item The 3rd algorithm uses the Sharpe ratio as the criterion to
  assess the significance of the estimated value $\bar X_t$ of $\E
  X_t$. Effectively we are treating the problem of hedging as one of
  trading, under the constraint that only short positions on SPY are
  allowed.

  To this end, we need not only an estimate of the mean value of $X_t$
  ($\E X_t$) but also its variance. Considering that we are always
  dealing with a short recent history of SPY, it is reasonable to
  assume the returns of SPY $(X_s)_{s = t-L, \dots, t-1}$ are
  independent and identically normally distributed. Under this
  assumption, we infer the distribution of $X_t$ using {\it Maximum
    Likelihood Estimate} (MLE). This is a standard procedure. We cover
  it in \S\ref{sec:mle}.

  Suppose by MLE we have obtained $\mu_t = \E X_t$ and
  $\sigma_t^2 = \var(X_t)$. Then, taking the risk-free interest rate
  as 0, the Sharpe ratio is given by $s_t = \mu_t/\sigma_t$. We short
  SPY if $s_t < -c_1 < 0$ and clear the short position if
  $s_t > c_1 > 0$. In all other situations, we leave the hedge
  position unchanged from the previous day.

\item The 4th algorithm assesses the significance of the estimated
  downward trend of SPY using statistical testing. In the context of
  hedging, we want to test the null-hypothesis
  \[
    \mathcal H_0:\quad
    \E X_t = 0.
  \]
  against the alternative
  \[
    \mathcal H_1:\quad
    \E X_t \neq 0.
  \]
  The test-statistic to use for this purpose is
  \[
    T = \bar X_t \sqrt{L(L - 1)} \left[
      \sum_{s=t-L}^{t-1} (X_s - \bar X_t)^2
    \right]^{-1/2}.
  \]
  Under the null hypothesis, $T$ is known to have $t$-distribution
  with $L-1$ degrees of freedom. Let $F(\cdot)$ denote the
  distribution function of $t$-distribution with $L-1$ degrees of
  freedom. Define the $p$-value of the test as
  \[
    p = \left\{
      \begin{array}{ll}
        F(T) & T < 0 \\
        1 - F(T) & T > 0
      \end{array}
    \right.
  \]
  We reject the null hypothesis if, for a pre-defined value $\alpha$,
  we find $p < \alpha$. The $p$-value is the probability of obtaining
  a sample that produces a test statistic even larger in absolute value
  than the observed value $T$ under the null hypothesis. Thus the
  smaller the p-value, the less likely the null hypothesis. When we
  reject the null hypothesis on condition $p < \alpha$, we are assured
  that the probability of falsely rejecting the null hypothesis is no
  more than $\alpha$.

  For the purpose of hedging, if we reject the null hypothesis and
  find $\bar X_t < 0$, we short SPY by the due amount; if instead we
  reject the null hypothesis but find $\bar X_t > 0$, we clear any
  short position on SPY. In all other situations, we leave the hedge
  position unchanged from the previous day.
  
\end{enumerate}

\section{Ordinary Least Squares}\label{sec:ols}
When a linear relation eq.\eqref{eq:ghtht} is assumed between two
random variables $Y_t$ and $X_t$, we like to find the regression
coefficient $a$ from a sample $(X_s, Y_s)_{s=t-L, \dots,
  t-1}$. Without further knowledge about the residual $\epsilon_t$, it
is plausible to find $a$ as
\begin{equation}
  \label{eq:joi90}
  a = \argmin_{\beta} \sum_{s=t-L}^{t-1} (Y_s - \beta X_s)^2
  =
  \left(
    \sum_{s=t-L}^{t-1} X_s^2
  \right)^{-1}
  \left(
    \sum_{s=t-L}^{t-1} X_s Y_s
  \right)
\end{equation}
That is, we find $a$ as the value that minimises the total squared
error between the returns of the portfolio $Y_t$ and the returns of
SPY $X_t$ multiplied by the regression coefficient.

One can see this method is very general. If we want to hedge the
portfolio against multiple market factors, we can follow the same
procedure. In this case, equation \eqref{eq:joi90} would become an
equation system with as many unknown variables as there are factors to
hedge against. However, the quadratic nature of eq. \eqref{eq:joi90}
is carried over to the scenario of multiple regression
coefficients. Analytic solutions are therefore usually possible.

Since the sample $(X_s, Y_s)_{s=t-L, \dots, t-1}$ is available at time
$t$, we can formulate $a_t$, the regression coefficient at time $t$ as
a function of $t$:
\[
  a_t = f(t) = \left(
    \sum_{s=t-L}^{t-1} X_s^2
  \right)^{-1}
  \left(
    \sum_{s=t-L}^{t-1} X_s Y_s
  \right)
\]
As discussed in \S\ref{sec:lm}, $a_t$ as given in the above formula
tells us the dollar amount of SPY to short if we are to make the
portfolio return uncorrelated to that of SPY. Of course, here we
assume $a_t > 0$, as is often the case.

\section{Maximum Likelihood Estimate}\label{sec:mle}
As detailed in \S\ref{sec:when}, we need to estimate the distribution
of the returns of SPY $X_t$ if we want to use the Sharpe ratio to decide
when to create a hedge position. The method of {\it Maximum Likelihood
Estimate} is a common approach to this problem. Assume we have a
sample $(X_s)_{s = t-L, \dots, t-1}$, which is available at time
$t$. Also assume the future return $X_t$ as well as this observed
sample is normally distributed with mean $\mu$ and variance
$\sigma^2$. Then it is our objective to estimate the parameters $\mu$
and $\sigma^2$. Let $\phi(x, \mu, \sigma)$ denote the probability
density function of the aforementioned distribution. The likelihood of
a particular pair of values for $(\mu, \sigma)$, given the sample
$(X_s)_{s = t-L, \dots, t-1}$, is the following
\[
  L(\mu, \sigma) = \prod_{s=t-L}^{t-1} \phi(X_s, \mu, \sigma)
\]
The MLE method aims to find the most likely values of the parameters
$(\mu, \sigma)$, provided that the sample $(X_s)_{s = t-L, \dots,
  t-1}$ has been observed. In other words, we would like to maximize
the function $L$ with respect to both of its arguments
$(\mu, \sigma)$. The values of $(\mu, \sigma)$ that maximize $L$ is
what we are looking for.

What remains to do is the standard procedure of optimization, for
which a variety of methods are available. In the particular case of
a normal distribution, the solution is simply
\begin{eqnarray*}
  \mu &=& {1 \over L} \sum_{s=t-L}^{t-1} X_s, \\
  \sigma^2 &=& {1 \over L} \sum_{s=t-L}^{t-1} (X_s - \mu)^2.
\end{eqnarray*}
That is, the MLE solution to $(\mu, \sigma)$ is indeed the sample mean
and the sample standard deviation without Bessel's correction.
While assuming a normal distribution for $X_t$ is convenient and often
sufficiently accurate as well, it may become inadequate when the
effects of heavy tails and heteroscedasticity are more pronounced. We
leave these topics to future research.

\section{Parameters and Performance of Hedge Algorithms}


\bibliographystyle{unsrt}
\bibliography{../../../kkasi/thesis/econophysics}
\end{document}
