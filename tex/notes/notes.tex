\documentclass{report}
\RequirePackage[colorlinks,citecolor=blue,urlcolor=blue,bookmarksopen=true]{hyperref}
\usepackage{bookmark}
\input{../physics_common}

\title{Playbook of Equity Trading}
\author{Xie Xiaolei}
\date{\today}

\begin{document}
\maketitle

\chapter{Order Execution}
\section{Order Types}
\begin{enumerate}
\item Market order - executed immdiately at the best available price from the order book.
  Different prices for different batches of the same order may be the result for large-volume orders.
  
\item Limit order - may or may not be executed. Must be set at a lower price than the market price
  for buy limit orders and at a price higher than the market price for sell limit orders.
  The executed price is the limit price or better.

  Conditions can be attached to a limit order:
  \begin{itemize}
  \item {\it fill-or-kill (FOK)}: The order must be executed immediately and completely or otherwise
    cancelled altogether.
  \item {\it all-or-none (AON)}: The order need not be executed immediately - a later time is OK - but it must
    be executed at its entirety. For a buy order, the order book must have enough shares on offer at prices equal to
    or lower than the limit price for the order to be carried through.
    Conversely, for a sell order, the order book must have enough bids at prices equal to or higher than the limit price.
  \end{itemize}

  Validity in time of an order can also be specified:
  \begin{itemize}
  \item {\it Good-till-canceled}: Valid until canceled.
  \item Valid until a specified time point, e.g. the end of the trading day.
  \end{itemize}

\item Stop order - think of stop loss.
  For a sell stop order, the order is triggered as soon as a transaction is observed at
  a price equal to or lower than the stop price; the order then becomes a market order.
  For a buy stop order, the order is triggered as soon as a transaction is observed at
  a price equal to or higher than the stop price.
  
\item Stop-limit order - similar to a stop order, but is converted to a limit order when triggered.

\end{enumerate}

\section{Execution Algorithms}
\begin{enumerate}
\item {\bf Volume Weighted Average Price}

  ``However, there are other uses for the VWAP, and one such strategy is to purchase a stock for individual investors just as the VWAP pierces the intraday VWAP moving average, as this can indicate a momentum shift in the share price.''- Investpedia

  ``Often, a VWAP trade will buy or sell 40\% of a trade in the first half of the day and then the other 60\% in the second half of the day''.    - Wikipedia
  
\item{\bf Time Weighted Average Price}

``A TWAP trade would most likely execute an even 50/50 volume in the first and second half of the day'' - Wikipedia

  
\end{enumerate}

\chapter{Market Impact \& Optimal Execution}
\section{Statistics of the order book}
Bouchaud et al \cite{BouchaudMezardPotters2002} gives a model of the
order book: let $a(t)$ be the lowest asking price and $b(t)$ the
highest bidding price at time $t$. The new buy limt order comes in at
price $b(t) - \Delta_b$ and the new sell limit order comes in at
$a(t) + \Delta_a$. The authors reported that $\Delta_a, \Delta_b$
could be considered independent, identically distributed. Let
$\Delta$ have the same distribution as $\Delta_a$ and $\Delta_b$.

The distribution of $\Delta$ has Pareto tail. Let
$f_{\Delta | \Delta \geq 1}$ denote the probability density function
of $\Delta$ conditional on $\Delta \geq 1$.
\begin{equation}
  \label{eq:rtwhyt}
  f_{\Delta | \Delta \geq 1}(x) = {
    K^\alpha
    \over
    (\mu + x)^{\alpha + 1}
  } \frac{1}{\P(\Delta \geq 1)}
\end{equation}
The tail index $\alpha$ is reported to be approximately $0.6$ for all
equities. Note this implies $\E \Delta = \infty$. It is also reported
that the distribution of $\log V$, $V$ being the size of the new limit
order, buy or sell, is uniformly distributed. Formally
\begin{equation}
  \label{eq:pojfg}
  \P(V \leq x) = a(\log x - b)
  \quad
  e^b \leq x \leq e^{1/a + b}
\end{equation}
The authors also reported, empirically
\begin{eqnarray}
  \E(V|\Delta = x, x \leq \Delta^*) &=& c \nonumber \\
  \E(V|\Delta = x, x > \Delta^*) &=& c' x^{-\nu},\; \nu \approx 1.5 \label{eq:rfepomj}
\end{eqnarray}
where $\Delta^*$ is a constant dependent on the stock itself.

From \eqref{eq:rtwhyt} and \eqref{eq:rfepomj}, one can derive the average
volume when $\Delta \leq \Delta^*$ and when $\Delta > \Delta^*$:
\begin{eqnarray*}
  \E(V \1{\Delta \leq \Delta^*}) &=& c \P(\Delta \leq \Delta^*) \\
  \E(V \1{\Delta > \Delta^*}) &=& \int_{\Delta^*}^\infty c' x^{
                                  -\alpha - \nu - 1
                                  } K^\alpha \left(
                                  \frac{\mu}{x} + 1
                                  \right)^{-\alpha - 1}
                                  \frac{dx}{\P(\Delta > 1)}
\end{eqnarray*}
By Karamata's theorem,
\[
  \E(V \1{\Delta > \Delta^*}) \sim c' \left(
    {\mu \over \Delta^*} + 1
  \right)^{-\alpha - 1} K^\alpha
  {(\Delta^*)^{-\alpha-\nu} \over \alpha + \nu}
  {1 \over \P(\Delta > 1)}
  \quad
  \text{ as }
  \Delta^* \to \infty
\]


\section{Square root law}
Bouchaud et al. \cite{Bouchaud2011} gave a model to the average price
change of an equity following the completion of a large order. It was
reported that the law was valid regardless of the way of execution -
over-the-counter or electronically, and regardless of the execution
strategy - using limit or market orders. Assume the order is executed
in its entirety within 1 trading day.
\begin{itemize}
\item $\{P_t\}_{t \in \reals_+}$: the stochastic process of prices of
  the equity.
\item $\tau$: time of the trade that fullfils the order. The first
  trade of the order is assumed to happen at a deterministic time
  $s < \tau$.
\item $Q$: size of the order, i.e. the number of shares that are
  intended to be traded - bought or sold.
\item $\{V_n\}_{n \in \naturals}$: Daily trading volumes of the
  equity.
\item $\{\sigma_n\}_{n \in \naturals}$: Daily volatilities of the
  equity.
\end{itemize}
Bouchaud et al. argued
\[
  \E (P_\tau - P_s) = c \E \sigma_{\ceil \tau} \left(
    {Q \over \E V_{\ceil \tau}}
  \right)^{1/2}
\]
The main arguments are as follows: Traders do not reveal their
intensions too early but rather send orders depending on the market
situation. Hence there is a latent order book that reflect the
interests of traders. Let
\begin{itemize}
\item $\lambda(u)$: the number of new shares added to the buy side of
  the latent order book at price $P_t - u$ in unit time and equally,
  the number of new shares added to the sell side of the latent order
  book at price $P_t + u$ in unit time. It is assumed that the buy
  side and the sell side of the latent order book change in the same
  way with the market price and $\Delta$ is independent of the market
  price.

\item $v(u, u')$: the number of shares on the buy side of the latent
  order book that are moved from price $P_t - u$ to price $P_t - u'$
  in unit time. Equally, it is also the number of shares on the sell
  side of the latent order book that are moved from price $P_t + u$
  to price $P_t + u'$.

\item $W (u, t)$: the volume on the latent order book for buy (sell)
  at price $P_t - u$ ($P_t + u$). $\rho(u, t) = \E W(u, t)$.
\end{itemize}

Essentially we assume that changes to the latent order book are always
symmetric and speculative in nature.




\chapter{Trading Strategies}
\section{volatility contraction \& breakouts}
Define for $k \in \naturals$
\begin{eqnarray*}
  H_k &=&   \sup_{k-1 < s \leq k} P_s \\
  L_k &=&   \inf_{k-1 < s \leq k} P_s 
\end{eqnarray*}
At time $t$, find the day $m$:
\[
  m = \max\left\{l \leq k \leq \floor t - 1:
  H_k \geq \max_{k < l \leq \floor t} H_l,
  \;
  L_k \leq \min_{k < l \leq \floor t} L_l
  \right\}
\]
A breakout happens at day $\ceil t$ if $L_{\ceil t} \geq H_m$ or
$H_{\ceil t} \leq L_m$. They indicate upward/downward trends
respectively.
The idea is that contracted volatility (as measured by the high-low
difference) followed by a breakout indicates the beginning of a trend.
The 3 quantities - volume, range, and $\var(\log P_t - \log P_{t-1})$
are expected to be highly dependent, as they all describe volatility.




\bibliographystyle{unsrt}
\bibliography{../../../kkasi/thesis/econophysics}
\end{document}
