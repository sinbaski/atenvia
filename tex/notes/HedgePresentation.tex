%%%%%%%%%%%%%%%%%%%%%%%%%%%%%%%%%%%%%%%%%
% Beamer Presentation
% LaTeX Template
% Version 1.0 (10/11/12)
%
% This template has been downloaded from:
% http://www.LaTeXTemplates.com
%
% License:
% CC BY-NC-SA 3.0 (http://creativecommons.org/licenses/by-nc-sa/3.0/)
%
%%%%%%%%%%%%%%%%%%%%%%%%%%%%%%%%%%%%%%%%%

%----------------------------------------------------------------------------------------
%	PACKAGES AND THEMES
%----------------------------------------------------------------------------------------

\documentclass{beamer}

\mode<presentation> {

% The Beamer class comes with a number of default slide themes
% which change the colors and layouts of slides. Below this is a list
% of all the themes, uncomment each in turn to see what they look like.

% \usetheme{default}
%\usetheme{AnnArbor}
%\usetheme{Antibes}
%\usetheme{Bergen}
% \usetheme{Berkeley}
%\usetheme{Berlin}
%\usetheme{Boadilla}
%\usetheme{CambridgeUS}
%\usetheme{Copenhagen}
%\usetheme{Darmstadt}
%\usetheme{Dresden}
%\usetheme{Frankfurt}
%\usetheme{Goettingen}
%\usetheme{Hannover}
%\usetheme{Ilmenau}
%\usetheme{JuanLesPins}
%\usetheme{Luebeck}
%\usetheme{Madrid}
%\usetheme{Malmoe}
%\usetheme{Marburg}
%\usetheme{Montpellier}
%\usetheme{PaloAlto}
%\usetheme{Pittsburgh}
%\usetheme{Rochester}
%\usetheme{Singapore}
%\usetheme{Szeged}
%\usetheme{Warsaw}

% As well as themes, the Beamer class has a number of color themes
% for any slide theme. Uncomment each of these in turn to see how it
% changes the colors of your current slide theme.

%\usecolortheme{albatross}
%\usecolortheme{beaver}
%\usecolortheme{beetle}
%\usecolortheme{crane}
%\usecolortheme{dolphin}
%\usecolortheme{dove}
%\usecolortheme{fly}
%\usecolortheme{lily}
%\usecolortheme{orchid}
%\usecolortheme{rose}
%\usecolortheme{seagull}
%\usecolortheme{seahorse}
%\usecolortheme{whale}
%\usecolortheme{wolverine}

%\setbeamertemplate{footline} % To remove the footer line in all slides uncomment this line
%\setbeamertemplate{footline}[page number] % To replace the footer
%line in all slides with a simple slide count uncomment this line

%\setbeamertemplate{navigation symbols}{} % To remove the navigation
%symbols from the bottom of all slides uncomment this line
}
%% \usepackage[dvipsnames]{xcolor}
\usepackage{graphicx} % Allows including images
\usepackage{booktabs} % Allows the use of \toprule, \midrule and \bottomrule
% in tables
\usepackage{xcolor}

\DeclareGraphicsExtensions{.pdf,.png,.jpg}
\usepackage{amssymb,amsmath,amsthm,amsfonts}
% \usepackage{mathrsfs}
% \usepackage{dsfont}
\usepackage{enumerate}
% \usepackage{etoolbox}

%% \newtheorem{mdef}{Definition}

\newcommand{\eqsplit}[2]{
  \begin{equation}\label{#2}
    \begin{split}
      #1
    \end{split}
  \end{equation}}
\newcommand{\eqnsplit}[1]{
  \begin{eqnarray*}
    #1
  \end{eqnarray*}}
\newcommand{\tran}[1]{
  \tilde{#1}
}
\newcommand{\td}[2]{
  \frac{d #1}{d #2}
}
%% \newcommand{\pd}[2]{
%%   \frac{\partial #1}{\partial #2}
%% }
\newcommand{\ppd}[2]{
  \frac{\partial^2 #1}{\partial #2^2}
}
\newcommand{\pdd}[3]{
  \frac{\partial^2 #1}{\partial #2 \partial #3}
}
\newcommand{\otd}[1]{
  \frac{d}{d #1}
}
\newcommand{\opd}[1]{
  \frac{\partial}{\partial #1}
}
\newcommand{\oppd}[1]{
  \frac{\partial^2}{\partial #1^2}
}
\newcommand{\opdd}[2]{
  \frac{\partial^2}{\partial #1 \partial #2}
}
\newcommand{\ket}[1]{
  |#1\rangle
}
\newcommand{\bra}[1]{
  \langle#1|
}
\newcommand{\inn}[1]{
  \langle#1\rangle
}
\newcommand{\mean}[1]{
  \langle#1\rangle
}
\newcommand{\tr}{
  \text{tr}\,
}
\newcommand{\re}{
  \text{Re}\,
}
\newcommand\im{
  \text{Im}\,
}
\newcommand{\arcsinh}{
  \sinh^{-1}
}
\newcommand{\arccosh}{
  \cosh^{-1}
}
\newcommand{\erfc}{
  \text{erfc}
}
\newcommand{\E}{
  \mathbb{E}
}
\renewcommand{\P}{
  \mathbb{P}
}
\newcommand{\I}{
  \mathbf{1}
}
\newcommand{\1}[1]{
  \mathbf{1}_{\{#1\}}
}
\newcommand{\diag}{\operatorname{diag}}
% \newcommand{\limsup}{\operatorname{limsup}}
%% \newcommand{\diag}{
%%   \text{diag\,}
%% }
\newcommand{\pop}[1]{
\varepsilon_{#1}
}

\newcommand{\M}{
  {\text{max}}
}
\newcommand{\m}{
  {\text{min}}
}

% \newcommand{\ph}{
%   {\text{arg}\,}
% }
\newcommand\erf{
  \text{erf}
}
\renewcommand\vec[1]{
  \mathbf{#1}
}
\newcommand\cov{
  \text{cov}
}
\newcommand\var{
  \text{var}
}
\newcommand\cor{
  \text{cor}
}
\DeclareMathOperator*{\argmin}{argmin}
\DeclareMathOperator*{\argmax}{argmax}
\newcommand\mat[1]{
  \mathbf{#1}
}
\newcommand\mtx[1]{
  \mathbf{#1}
}
\renewcommand \L {
  \mathfrak{leb}
}
\newcommand{\floor}[1]{
  \lfloor #1\rfloor
}
\newcommand{\ceil}[1]{
  \lceil #1\rceil
}
\newcommand{\supp} {
  \text{supp}\,
}
\newcommand{\img} {
  \text{Img}\,
}
\newcommand{\reals} {
  \mathbb R                 
}
\newcommand{\integers} {
  \mathbb Z
}
\newcommand{\naturals} {
  \mathbb N
}
\newcommand{\sphere} {
  \mathbb S
}
\newtheorem{case}{case}

\providecommand{\keywords}[1]{\textbf{\textit{Keywords: }} #1}
\providecommand{\subjclass}[1]{\textbf{\textit{Subject class: }} #1}

\newcommand{\bfth}{\mbox{\boldmath$\theta$}}
\newcommand{\bfTh}{\mbox{\boldmath$\Theta$}}
\newcommand{\bfmu}{\mbox{\boldmath$\mu$}}
\newcommand{\bfrho}{\mbox{\boldmath$\rho$}}
\newcommand{\bfvarphi}{\mbox{\boldmath$\varphi$}}
\newcommand{\bfbeta}{\mbox{\boldmath$\beta$}}
\newcommand{\bfalpha}{\mbox{\boldmath$\alpha$}}
\newcommand{\bfeta}{\mbox{\boldmath$\eta$}}
\newcommand{\bfsigma}{\mbox{\boldmath$\sigma$}}
\newcommand{\bfzeta}{\mbox{\boldmath$\zeta$}}
\newcommand{\bfxi}{\mbox{\boldmath$\xi$}}

\newcommand{\stj}{\stackrel{J_1}{\rightarrow}}
\newcommand{\stv}{\stackrel{v}{\rightarrow}}
\newcommand{\stw}{\stackrel{w}{\rightarrow}}
\newcommand{\stww}{\stackrel{\wh w}{\rightarrow}}
\newcommand{\ep}{\epsilon}
\newcommand{\vap}{\varphi}
\newcommand{\La}{\Lambda}

%% \iftoggle{theoremdefs}{
%% \newtheorem{theorem}{Theorem}
%% \newtheorem{acknowledgement}{Acknowledgement}
%% \newtheorem{axiom}{Axiom}
%% \newtheorem{claim}{Claim}
%% \newtheorem{hypothesis}{Hypothesis}
%% \newtheorem{conclusion}{Conclusion}
%% \newtheorem{condition}{Condition}
%% \newtheorem{conjecture}{Conjecture}
%% \newtheorem{corollary}{Corollary}
%% \newtheorem{criterion}{Criterion}
%% \newtheorem{definition}{Definition}
%% \newtheorem{example}{Example}
%% \newtheorem{exercise}{Exercise}
%% \newtheorem{lemma}{Lemma}
%% \newtheorem{notation}{Notation}
%% \newtheorem{problem}{Problem}
%% \newtheorem{proposition}{Proposition}
%% \newtheorem{remark}{Remark}
%% \newtheorem{solution}{Solution}
%% \newtheorem{summary}{Summary}
%% }{
%% }


%----------------------------------------------------------------------------------------
%	TITLE PAGE
%----------------------------------------------------------------------------------------

\title{Hedge Algorithms}
% The short title appears at the bottom of
% every slide, the full title is only
% on the title page

\author{Xie Xiaolei} % Your name
\institute[UCPH] % Your institution as it will appear on the bottom of every slide, may be shorthand to save space
{
Atenvia Asset Management \\ % Your institution for the title page
\medskip
\textit{xiaolei.xie@atenvia.com} % Your email address
}
\date{\today} % Date, can be changed to a custom date

\begin{document}

\begin{frame}
\titlepage % Print the title page as the first slide
\end{frame}

% \begin{frame}
% \frametitle{Overview}
% \tableofcontents
% \end{frame}

%----------------------------------------------------------------------------------------
%	PRESENTATION SLIDES
%----------------------------------------------------------------------------------------
% \section{Simple dependence models}
%------------------------------------------------
\begin{frame}
  \frametitle{Agenda}
  \begin{itemize}
  \item How much to hedge
  \item When to hedge
  \end{itemize}
\end{frame}

\section{How much to hedge}
\begin{frame}
  \frametitle{Linear Model of Asset Returns}
  \begin{eqnarray*}
    Y_t &:& \text{Portfolio Returns} \\
    X_t &:& \text{SPY Returns}
  \end{eqnarray*}
  A linear relation often exists
  \[
    Y_t = a X_t + \epsilon_t  
  \]
  This gives a hedging algorithm:
  \[
    Y_t - a X_t = \epsilon_t    
  \]
  \begin{itemize}
  \item for every dollar worth of the portfolio, we short $a$ dollars worth of SPY
  \item the return of our holding is independent of $X_t$.
  \end{itemize}
\end{frame}

\begin{frame}
  \frametitle{Ordinary Least Squares}
  To calculate the regression coefficient $a$, we use {\it ordinary
    least squares}.
  \begin{equation*}
    a = \argmin_{\beta} \sum_{s=t-L}^{t-1} (Y_s - \beta X_s)^2
    =
    \left(
      \sum_{s=t-L}^{t-1} X_s^2
    \right)^{-1}
    \left(
      \sum_{s=t-L}^{t-1} X_s Y_s
    \right)
  \end{equation*}
  $a$ should be assigned the value that minimises the total squared
  error between the returns of the portfolio $Y_t$ and the returns of
  SPY $X_t$ multiplied by the regression coefficient.
\end{frame}

\section{When to hedge}
\begin{frame}
  \frametitle{When to hedge}
  \begin{itemize}
  \item Hedge when a downward trend dominates the motion - when the
    return of SPY ($X_t$) has a negative mean ($\E X_t < 0$). Estimating
    $\E X_t$ is straightforward:
    \[
      \bar X_t = {1 \over L} \sum_{i=t-L}^{t-1} X_i \overset{a.s.}{\to} \E X_t,
      \quad
      L \to \infty
    \]
  \item If we hedge every time $\bar X_t$ is found negative, we'll have
    an erratic behaviour.
  \item We want to hedge only if the estimated value of $\E X_t$ is
    significant.
  \item How to assess the significance of the estimate? Each approach
    leads to a hedge algorithm.
  \end{itemize}
\end{frame}

\begin{frame}
  \frametitle{Significance assessment based on drawdown}
  \begin{itemize}
  \item When the drawdown exceeds a value $c_1$, we hedge by the due
    amount;
  \item When the portfolio recovers and the drawdown is below a value
    $c_2 < c_1$, we lift the hedge;
  \item When the amount of change to the hedge position is small,
    we ignore the change to reduce the cost of trading due to
    hedging.
  \end{itemize}
\end{frame}

\begin{frame}
  \frametitle{Significance assessment based on annualised equity return}
  \begin{itemize}
  \item Compare $\bar X_t$ with a constant threshold $c_1 > 0$ which
    derives from typical annualised equity returns.
  \item If $\bar X_t < -c_1$, we hedge;
  \item if $\bar X_t > c_1$, we lift the hedge;
  \item otherwise, we leave the hedge unchanged from the previous day
  \end{itemize}
\end{frame}

\begin{frame}
  \frametitle{Significance assessment based on Sharpe Ratio}
  \begin{itemize}
  \item Hedging treated as trading
  \item Only short positions allowed
  \end{itemize}
  Compute the Sharpe ratio as
  \[
    s_t = {\E X_t \over \sqrt{\var(X_t)}}
    \quad
    \text{risk-free interest rate taken as 0}
  \]
  To compute $\var(X_t)$, we need the distribution of $X_t$.
  \begin{itemize}
  \item Assume $(X_s)_{s=t-L,\dots, t-1}$ are independent, identically
    normally distributed with mean $\mu$ and variance $\sigma^2$.
    ($X_t \sim N(\mu, \sigma^2)$).
  \item Estimate $(\mu, \sigma)$ by {\it Maximum Likelihood
      Estimate}. The likelihood of a particular pair of values for
    $(\mu, \sigma)$, given the sample $(X_s)_{s = t-L, \dots, t-1}$,
    is
    \[
      L(\mu, \sigma) = \prod_{s=t-L}^{t-1} \phi(X_s, \mu, \sigma)
    \]
    where $\phi(x, \mu, \sigma)$ is the probability density function
    of $N(\mu, \sigma^2)$.
  \end{itemize}
\end{frame}

\begin{frame}
  \frametitle{Significance assessment based on Sharpe Ratio}
  We use standard optimization algorithms to find $(\mu, \sigma)$ that
  maximise $L(\mu, \sigma)$. For a normal distribution:
\begin{eqnarray*}
  \mu &=& {1 \over L} \sum_{s=t-L}^{t-1} X_s, \\
  \sigma^2 &=& {1 \over L} \sum_{s=t-L}^{t-1} (X_s - \mu)^2.
\end{eqnarray*}
\begin{itemize}
\item a normal distribution is good enough in most cases
\item Such effects as volatility clustering and heavy tails are not
  accounted for by a normal distribution.
\item We might need to use GARCH models and e.g. $t$-distributions in
  these intricate situations.
\end{itemize}
\end{frame}

\begin{frame}
  \frametitle{Significance assessment based on Sharpe Ratio}
  Once $(\mu, \sigma)$ has been estimated, we
  \begin{itemize}
  \item short SPY if $s_t < -c_1 < 0$
  \item clear the short position if $s_t > c_1 > 0$
  \item leave the hedge position unchanged from the previous day in
    all other cases
  \end{itemize}
\end{frame}

\begin{frame}
  \frametitle{Significance assessment based on hypothesis testing}
  We want to test the null-hypothesis
  \[
    \mathcal H_0:\quad
    \E X_t = 0.
    \quad
    \text{against }
    \quad
    \mathcal H_1:\quad
    \E X_t \neq 0.
  \]
  The test-statistic to use for this purpose is
  \[
    T = \bar X_t \sqrt{L(L - 1)} \left[
      \sum_{s=t-L}^{t-1} (X_s - \bar X_t)^2
    \right]^{-1/2}.
  \]
  Under the null hypothesis, $T$ is known to have $t$-distribution
  with $L-1$ degrees of freedom.
\end{frame}

\begin{frame}
  \frametitle{Significance assessment based on hypothesis testing}
  $F(\cdot)$: distribution function of $t$-distribution with $L-1$
  degrees of freedom. Define the $p$-value of the test as
  \[
    p = \left\{
      \begin{array}{ll}
        F(T) & T < 0 \\
        1 - F(T) & T > 0
      \end{array}
    \right.
  \]
  For a defined value $\alpha$,
  \begin{itemize}
  \item Reject the null hypothesis if $p < \alpha$
  \item The $p$-value is the probability of obtaining a sample that
    produces a test statistic even larger in absolute value than the
    observed value $T$ under the null hypothesis
  \item The smaller the p-value, the less likely the null
    hypothesis.
  \item When we reject the null hypothesis on condition $p < \alpha$,
    we are assured that the probability of falsely rejecting the null
    hypothesis is no more than $\alpha$.
  \end{itemize}
\end{frame}

\begin{frame}
  \frametitle{Significance assessment based on hypothesis testing}
  \begin{itemize}
  \item If we reject the null hypothesis and $\bar X_t < 0$, we short
    SPY;
  \item If we reject the null hypothesis and $\bar X_t > 0$, we clear
    the hedge position;
  \item Otherwise we leave the hedge position unchanged from the
    previous day.
  \end{itemize}
\end{frame}

\begin{frame}
  \frametitle{Performance of the Algorithms}
  \begin{table}[htb!]
    \centering
    \begin{tabular}{|c|c|c|c|c|}
      \hline
      & max DD & mean DD & max hit & mean hit \\
      \hline
      1st & 10.9\% &  0.6\% &  10.1\% &  3.2\% \\
      \hline
      2nd & 5.8\% &   0.5\% &  14.6\% &  8.1\% \\
      \hline
      3rd & 9.9\% &   0.6\% &  3.8\% &   0.0\% \\
      \hline
      4th & 9.5\% &   0.6\% &  2.4\% &  -1.4\% \\
      \hline
    \end{tabular}
    \caption{Drawdown and Hit}
    \label{tab:dd_hit}
  \end{table}
\end{frame}

\begin{frame}
  \frametitle{Performance of the Algorithms}
  \begin{minipage}{0.8\linewidth}
    \begin{figure}[htb!]
      \centering
      \includegraphics[width=1.0\textwidth]{HedgeCurve}
      \caption{Equity curves with/without the hedge}
      \label{fig:hedge_performance}
    \end{figure}
  \end{minipage}\hfill
  \begin{minipage}[t]{0.15\linewidth}
    \textcolor[HTML]{000000}{\small Original}
    \begin{itemize}
    \item \textcolor[HTML]{FF0000}{1st}
    \item \textcolor[HTML]{0000FF}{2nd}
    \item \textcolor[HTML]{00FF00}{3rd}
    \item \textcolor[HTML]{6600CC}{4th}
    \end{itemize}
  \end{minipage}
\end{frame}


\begin{frame}
   \frametitle{Thank you!}
   Questions?
 \end{frame}
\bibliographystyle{unsrt}
\bibliography{../../thesis/econophysics}
\end{document} 
